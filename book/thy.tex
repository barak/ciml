\chapter{Learning Theory} \label{sec:thy}

\chapterquote{The Universe is under no obligation to make sense to you.}{Neil~deGrasse~Tyson}

\begin{learningobjectives}
\item Explain why inductive bias is necessary.
\item Define the PAC model and explain why both the ``P'' and ``A''
  are necessary.
\item Explain the relationship between complexity measures and
  regularizers.
\item Identify the role of complexity in generalization.
\item Formalize the relationship between margins and complexity.
\end{learningobjectives}

\dependencies{}

\newthought{By now, you are an expert} at building learning
algorithms.  You probably understand how they work, intuitively.  And
you understand why they should generalize.  However, there are several
basic questions you might want to know the answer to.  Is learning
always possible?  How many training examples will I need to do a good
job learning?  Is my test performance going to be much worse than my
training performance?  The key idea that underlies all these answer is
that \emph{simple functions generalize well.}

The amazing thing is that you can actually prove strong results that
address the above questions.  In this chapter, you will learn some of
the most important results in learning theory that attempt to answer
these questions.  The goal of this chapter is not theory for theory's
sake, but rather as a way to better understand why learning models
work, and how to use this theory to build better algorithms.  As a
concrete example, we will see how $2$-norm regularization provably
leads to better generalization performance, thus justifying our common
practice!

\section{The Role of Theory}

In contrast to the quote at the start of this chapter, a practitioner
friend once said ``I would happily give up a few percent performance
for an algorithm that I can understand.''  Both perspectives are
completely valid, and are actually not contradictory.  The second
statement is presupposing that theory helps you understand, which
hopefully you'll find to be the case in this chapter.

Theory can serve two roles.  It can justify and help understand why
common practice works.  This is the ``theory after'' view.  It can
also serve to suggest new algorithms and approaches that turn out to
work well in practice.  This is the ``theory before'' view.  Often, it
turns out to be a mix.  Practitioners discover something that works
surprisingly well.  Theorists figure out why it works and prove
something about it.  And in the process, they make it better or find
new algorithms that more directly exploit whatever property it is that
made the theory go through.

Theory can also help you understand what's possible and \emph{what's
  not possible.}  One of the first things we'll see is that, in
general, machine learning can not work.  Of course it \emph{does}
work, so this means that we need to think harder about what it means
for learning algorithms to work.  By understanding what's not
possible, you can focus our energy on things that are.

Probably the biggest \emph{practical} success story for theoretical
machine learning is the theory of \concept{boosting}, which you won't
actually see in this chapter.  (You'll have to wait for
Chapter~\ref{sec:ens}.)  Boosting is a very simple style of algorithm
that came out of theoretical machine learning, and has proven to be
incredibly successful in practice.  So much so that it is one of the
de facto algorithms to run when someone gives you a new data set.  In
fact, in 2004, Yoav Freund and Rob Schapire won the ACM's Paris
Kanellakis Award for their boosting algorithm AdaBoost.  This award is
given for theoretical accomplishments that have had a significant and
demonstrable effect on the practice of computing.\sidenote{In 2008,
  Corinna Cortes and Vladimir Vapnik won it for support vector
  machines.}

\section{Induction is Impossible}

One nice thing about theory is that it forces you to be \emph{precise}
about what you are trying to do.  You've already seen a formal
definition of binary classification in Chapter~\ref{sec:complex}.  But
let's take a step back and re-analyze what it means to learn to do
binary classification.

From an \emph{algorithmic} perspective, a natural question is whether
there is an ``ultimate'' learning algorithm, $\cA^{\text{awesome}}$,
that solves the Binary Classification problem above.  In other words,
have you been wasting your time learning about $K$NN and Perceptron
and decision trees, when $\cA^{\text{awesome}}$ is out there.

What would such an ultimate learning algorithm do?  You would like it
to take in a data set $D$ and produce a function $f$.  No matter what
$D$ looks like, this function $f$ should get perfect classification on
all future examples drawn from the same distribution that produced
$D$.

A little bit of introspection should demonstrate that this is
impossible.  For instance, there might be label noise in our
distribution.  As a very simple example, let $\cX = \{-1,+1\}$ (i.e.,
a one-dimensional, binary distribution.  Define the data distribution
as:
\begin{align}  \label{eq:thy:simpledist}
\cD( \langle +1 \rangle , +1 ) &= 0.4 &
\cD( \langle -1 \rangle , -1 ) &= 0.4 \\
\cD( \langle +1 \rangle , -1 ) &= 0.1 &
\cD( \langle -1 \rangle , +1 ) &= 0.1
\end{align}
In other words, $80\%$ of data points in this distribution have $x = y$
and $20\%$ don't.  No matter what function your learning algorithm
produces, there's no way that it can do better than $20\%$ error on
this data.

\thinkaboutit{It's clear that if your algorithm produces a
  \emph{deterministic} function that it cannot do better than $20\%$
  error.  What if it produces a stochastic (aka randomized) function?}

Given this, it seems hopeless to have an algorithm
$\cA^{\text{awesome}}$ that always achieves an error rate of zero.
The best that we can hope is that the error rate is not ``too large.''

Unfortunately, simply weakening our requirement on the error rate is
not enough to make learning possible.  The second source of difficulty
comes from the fact that the only access we have to the data
distribution is through sampling.  In particular, when trying to learn
about a distribution like that in~\ref{eq:thy:simpledist}, you only get
to see data points \emph{drawn} from that distribution.  You know that
``eventually'' you will see enough data points that your sample is
representative of the distribution, but it might not happen
immediately.  For instance, even though a fair coin will come up heads
only with probability $1/2$, it's completely plausible that in a
sequence of four coin flips you never see a tails, or perhaps only see
one tails.

So the second thing that we have to give up is the hope that
$\cA^{\text{awesome}}$ will \emph{always} work.  In particular, if we
happen to get a lousy sample of data from $\cD$, we need to allow
$\cA^{\text{awesome}}$ to do something completely unreasonable.

Thus, we cannot hope that $\cA^{\text{awesome}}$ will do perfectly,
every time.  We cannot even hope that it will do pretty well, all of
the time.  Nor can we hope that it will do perfectly, most of the
time.  The best \emph{best} we can reasonably hope of
$\cA^{\text{awesome}}$ is that it it will do pretty well, most of the
time.

\section{Probably Approximately Correct Learning}

\concept{Probably Approximately Correct} (\concept{PAC}) learning is a
formalism of inductive learning based on the realization that the best
we can hope of an algorithm is that it does a good job (i.e., is
approximately correct), most of the time (i.e., it is \emph{probably}
approximately correct).\sidenote{Leslie Valiant invented the notion of
  PAC learning in 1984.  In 2011, he received the Turing Award, the
  highest honor in computing for his work in learning theory,
  computational complexity and parallel systems.}

Consider a hypothetical learning algorithm.  You run it on ten
different binary classification data sets.  For each one, it comes
back with functions $f_1$, $f_2$, $\dots$, $f_{10}$.  For some reason,
whenever you run $f_4$ on a test point, it crashes your computer.  For
the other learned functions, their performance on test data is always
at most $5\%$ error.  If this situation is guaranteed to happen, then
this hypothetical learning algorithm is a PAC learning algorithm.  It
satisfies ``probably'' because it only failed in one out of ten cases,
and it's ``approximate'' because it achieved low, but non-zero, error
on the remainder of the cases.

This leads to the formal definition of an $(\ep,\de)$ PAC-learning
algorithm.  In this definition, $\ep$ plays the role of measuring
accuracy (in the previous example, $\ep = 0.05$) and $\de$ plays the
role of measuring failure (in the previous, $\de = 0.1$).

\begin{definition}
  An algorithm $\cA$ is an \textcolor{darkblue}{$(\ep,\de)$-PAC
    learning algorithm} if, \emph{for all} distributions $\cD$: given
  samples from $\cD$, the probability that it returns a ``bad
  function'' is at most $\de$; where a ``bad'' function is one with
  test error rate more than $\ep$ on $\cD$.
\end{definition}

There are two notions of \emph{efficiency} that matter in PAC
learning.  The first is the usual notion of \emph{computational
  complexity}.  You would prefer an algorithm that runs quickly to one
that takes forever.  The second is the notion of \concept{sample
  complexity}: the number of examples required for your algorithm to
achieve its goals.  Note that the goal of both of these measure of
complexity is to bound how much of a scarce resource your algorithm
uses.  In the computational case, the resource is CPU cycles.  In the
sample case, the resource is labeled examples.
\\
~\\
\noindent
{\bf Definition:} An algorithm $\cA$ is an
\textcolor{darkblue}{efficient $(\ep,\de)$-PAC learning algorithm} if
it is an $(\ep,\de)$-PAC learning algorithm whose runtime is
polynomial in $\frac 1 \ep$ and $\frac 1 \de$.

In other words, suppose that you want your algorithm to achieve $4\%$
error rate rather than $5\%$.  The runtime required to do so should not
go up by an exponential factor.

\section{PAC Learning of Conjunctions}

To get a better sense of PAC learning, we will start with a completely
irrelevant and uninteresting example.  The purpose of this example is
\emph{only} to help understand how PAC learning works.

The setting is learning conjunctions.  Your data points are binary
vectors, for instance $\vx = \langle 0,1,1,0,1 \rangle$.  Someone
guarantees for you that there is some boolean conjunction that defines
the true labeling of this data.  For instance, $x_1 \land \lnot x_2
\land x_5$ (``or'' is not allowed).  In formal terms, we often call
the true underlying classification function the \concept{concept}.  So
this is saying that the concept you are trying to learn is a
conjunction.  In this case, the boolean function would assign a
negative label to the example above.

Since you know that the concept you are trying to learn is a
conjunction, it makes sense that you would represent your function as
a conjunction as well.  For historical reasons, the function that you
learn is often called a \concept{hypothesis} and is often denoted $h$.
However, in keeping with the other notation in this book, we will
continue to denote it $f$.

Formally, the set up is as follows.  There is some distribution
$\cD^X$ over binary data points (vectors) $\vx = \langle x_1, x_2,
\dots, x_D \rangle$.  There is a fixed concept conjunction $c$ that we
are trying to learn.  There is no noise, so for any example $\vx$, its
true label is simply $y = c(\vx)$.

\TableSize{large}{thy:booleandata}%
  {Data set for learning conjunctions.}%
  {c|cccc}{
$y$ & $x_1$  & $x_2$ & $x_3$ & $x_4$   \\
\hline
+1 & 0 & 0 & 1 & 1 \\
+1 & 0 & 1 & 1 & 1 \\
-1 & 1 & 1 & 0 & 1
}

What is a reasonable algorithm in this case?  Suppose that you observe
the data set in Table~\ref{tab:thy:booleandata}.  From the first
example, we know that the true formula cannot include the term $x_1$.
If it did, this example would have to be negative, which it is not.
By the same reasoning, it cannot include $x_2$.  By analogous
reasoning, it also can neither include the term $\lnot x_3$ nor the
term $\lnot x_4$.

\newalgorithm%
  {thy:binconj}%
  {\FUN{BinaryConjunctionTrain}(\VAR{$\mat D$})}%
  {
\SETST{$f$}{$x_1 \land \lnot x_1 \land x_2 \land \lnot x_2 \land \cdots \land x_D \land \lnot x_D$}
  \COMMENT{initialize function}
\FORALL{positive examples $($\VAR{$\vx$},$\CON{+1})$ in \VAR{$\mat D$}}
\FOR{\VAR{d} = \CON{1} \dots \VAR{D}}
\IF{\VAR{$x_d$} = \CON{0}}
\SETST{f}{\VAR{$f$} without term ``\VAR{$x_d$}''}
\ELSE
\SETST{f}{\VAR{$f$} without term ``$\lnot$\VAR{$x_d$}''}
\ENDIF
\ENDFOR
\ENDFOR
\RETURN{\VAR{$f$}}
}

This suggests the algorithm in Algorithm~\ref{alg:thy:binconj},
colloquially the ``Throw Out Bad Terms'' algorithm.  In this algorithm,
you begin with a function that includes all possible $2D$ terms.  Note
that this function will initially classify everything as negative.
You then process each example in sequence.  On a negative example, you
do nothing.  On a positive example, you throw out terms from $f$ that
contradict the given positive example.
%
\thinkaboutit{Verify that Algorithm~\ref{alg:thy:binconj} maintains an
  \emph{invariant} that it always errs on the side of classifying
  examples negative and never errs the other way.}

%
If you run this algorithm on the data in
Table~\ref{tab:thy:booleandata}, the sequence of $f$s that you cycle
through are:
%
\begin{align}
f^0(\vx) &=  x_1 \land \lnot x_1 \land x_2 \land \lnot x_2 \land x_3 \land \lnot x_3 \land x_4 \land \lnot x_4 \\
f^1(\vx) &=  \lnot x_1 \land \lnot x_2 \land x_3 \land x_4 \\
f^2(\vx) &=  \lnot x_1 \land x_3 \land x_4 \\
f^3(\vx) &=  \lnot x_1 \land x_3 \land x_4
\end{align}
%
The first thing to notice about this algorithm is that after
processing an example, it is guaranteed to classify that example
correctly.  This observation \emph{requires} that there is no noise in
the data.

The second thing to notice is that it's very computationally
efficient.  Given a data set of $N$ examples in $D$ dimensions, it
takes $\cO(ND)$ time to process the data.  This is linear in the size
of the data set.

However, in order to be an efficient $(\ep,\de)$-PAC learning
algorithm, you need to be able to get a bound on the \concept{sample
  complexity} of this algorithm.  Sure, you know that its run time is
linear in the number of examples $N$.  But \emph{how many} examples $N$
do you need to see in order to guarantee that it achieves an error
rate of at most $\ep$ (in all but $\de$-many cases)?  Perhaps $N$ has
to be gigantic (like $2^{2^{D/\ep}}$) to (probably) guarantee a small
error.

The goal is to prove that the number of samples $N$ required to
(probably) achieve a small error is not-too-big.  The general proof
technique for this has essentially the same flavor as almost every PAC
learning proof around.  First, you define a ``bad thing.''  In this
case, a ``bad thing'' is that there is some term (say $\lnot x_8$)
that should have been thrown out, but wasn't.  Then you say: well, bad
things happen.  Then you notice that if this bad thing happened, you
must not have seen any positive training examples with $x_8=0$.  So
example with $x_8=0$ must have low probability (otherwise you would
have seen them).  So bad things must not be that common.

\begin{theorem} \label{thm:thy:conj}
  With probability at least $(1-\de)$: Algorithm~\ref{alg:thy:binconj}
  requires at most $N = \dots$ examples to achieve an error rate $\leq
  \ep$.
\end{theorem}

\begin{myproof}{\ref{thm:thy:conj}}
  Let $c$ be the concept you are trying to learn and let $\cD$ be the
  distribution that generates the data.

%  \begin{align}
%    \Pr_{\vx \sim \cD}\left[ c(\vx) \neq f(\vx) \right]
%    &= \Pr_{\vx \sim \cD} \left[

  A learned function $f$ can make a mistake if it contains \emph{any}
  term $t$ that is not in $c$.  There are initially $2D$ many terms in
  $f$, and any (or all!) of them might not be in $c$.  We want to
  ensure that the probability that $f$ makes an error is at most
  $\ep$.  It is sufficient to ensure that

For a term $t$
  (e.g., $\lnot x_5$), we say that $t$ ``negates'' an example $\vx$ if
  $t(\vx) = 0$.  Call a term $t$ ``bad'' if (a) it does not appear in
  $c$ and (b) has probability at least $\ep/2D$ of appearing (with
  respect to the unknown distribution $\cD$ over data points).

  First, we show that if we have no bad terms left in $f$, then $f$
  has an error rate at most $\ep$.

We know that $f$ contains \emph{at
    most} $2D$ terms, since it begins with $2D$ terms and throws them
  out.

  The algorithm begins with $2D$ terms (one for each variable and one
  for each negated variable).  Note that $f$ will only make one type
  of error: it can call positive examples negative, but can never call
  a negative example positive.  Let $c$ be the true concept (true
  boolean formula) and call a term ``bad'' if it does not appear in
  $c$.  A specific bad term (e.g., $\lnot x_5$) will cause $f$ to err
  only on positive examples that contain a corresponding bad value
  (e.g., $x_5 = 1$).    TODO... finish this
\end{myproof}

What we've shown in this theorem is that: \emph{if} the true
underlying concept is a boolean conjunction, \emph{and} there is no
noise, \emph{then} the ``Throw Out Bad Terms'' algorithm needs $N \leq
\dots$ examples in order to learn a boolean conjunction that is
$(1-\de)$-likely to achieve an error of at most $\ep$.  That is to
say, that the \concept{sample complexity} of ``Throw Out Bad Terms''
is $\dots$.  Moreover, since the algorithm's runtime is linear in $N$,
it is an efficient PAC learning algorithm.

\section{Occam's Razor: Simple Solutions Generalize}

The previous example of boolean conjunctions is mostly just a warm-up
exercise to understand PAC-style proofs in a concrete setting.  In
this section, you get to generalize the above argument to a much
larger range of learning problems.  We will still assume that there is
no noise, because it makes the analysis much simpler.  (Don't worry:
noise will be added eventually.)

William of Occam (c. 1288 -- c. 1348) was an English friar and
philosopher is is most famous for what later became known as Occam's
razor and popularized by Bertrand Russell.  The principle basically
states that you should only assume as much as you need.  Or, more
verbosely, ``if one can explain a phenomenon without assuming this or
that hypothetical entity, then there is no ground for assuming it
i.e. that one should always opt for an explanation in terms of the
fewest possible number of causes, factors, or variables.''  What Occam
actually wrote is the quote that began this chapter.

In a machine learning context, a reasonable paraphrase is ``simple
solutions generalize well.''  In other words, you have $10,000$
features you could be looking at.  If you're able to explain your
predictions using just $5$ of them, or using all $10,000$ of them,
then you should just use the $5$.

The Occam's razor theorem states that this is a good idea,
theoretically.  It essentially states that if you are learning some
unknown concept, and if you are able to fit your training data
perfectly, but you don't need to resort to a huge class of possible
functions to do so, then your learned function will generalize well.
It's an amazing theorem, due partly to the simplicity of its proof.
In some ways, the proof is actually \emph{easier} than the proof of
the boolean conjunctions, though it follows the same basic argument.

In order to state the theorem explicitly, you need to be able to think
about a \concept{hypothesis class}.  This is the set of possible
hypotheses that your algorithm searches through to find the ``best''
one.  In the case of the boolean conjunctions example, the hypothesis
class, $\cH$, is the set of all boolean formulae over $D$-many
variables.  In the case of a perceptron, your hypothesis class is the
set of all possible linear classifiers.  The hypothesis class for
boolean conjunctions is \emph{finite}; the hypothesis class for linear
classifiers is \emph{infinite}.  For Occam's razor, we can only work
with finite hypothesis classes.


\begin{theorem}[Occam's Bound] \label{thm:thy:occam}
  Suppose $\cA$ is an algorithm that learns a function $f$ from some
  \emph{finite} hypothesis class $\cH$.  Suppose the learned function
  always gets zero error on the training data.  Then, the sample
  complexity of $f$ is at most $\log \card{\cH}$.
\end{theorem}

TODO COMMENTS

\begin{myproof}{\ref{thm:thy:occam}}
  TODO
\end{myproof}

This theorem applies directly to the ``Throw Out Bad Terms''
algorithm, since (a) the hypothesis class is finite and (b) the
learned function always achieves zero error on the training data.  To
apply Occam's Bound, you need only compute the size of the hypothesis
class $\cH$ of boolean conjunctions.  You can compute this by noticing
that there are a total of $2D$ possible terms in any formula in
$\cH$.  Moreover, each term may or may not be in a formula.  So there
are $2^{2D} = 4^D$ possible formulae; thus, $\card{\cH} = 4^D$.
Applying Occam's Bound, we see that the sample complexity of this
algorithm is $N \leq \dots$.

Of course, Occam's Bound is general enough to capture other learning
algorithms as well.  In particular, it can capture decision trees!  In
the no-noise setting, a decision tree will always fit the training
data perfectly.  The only remaining difficulty is to compute the size
of the hypothesis class of a decision tree learner.

\TODOFigure{thy:dt}{picture of full decision tree}

For simplicity's sake, suppose that our decision tree algorithm always
learns complete trees: i.e., every branch from root to leaf is length
$D$.  So the number of split points in the tree (i.e., places where a
feature is queried) is $2^{D-1}$.  (See Figure~\ref{fig:thy:dt}.)
Each split point needs to be assigned a feature: there $D$-many
choices here.  This gives $D 2^{D-1}$ trees.  The last thing is that
there are $2^D$ leaves of the tree, each of which can take two
possible values, depending on whether this leaf is classified as $+1$
or $-1$: this is $2 \times 2^D = 2^{D+1}$ possibilities.  Putting this
all together gives a total number of trees $\card{\cH} = D 2^{D-1}
2^{D+1} = D 2^{2D} = D 4^D$.  Applying Occam's Bound, we see that
$TODO$ examples is enough to learn a decision tree!

\section{Complexity of Infinite Hypothesis Spaces}

Occam's Bound is a fantastic result for learning over \emph{finite}
hypothesis spaces.  Unfortunately, it is completely useless when
$\card{\cH} = \infty$.  This is because the proof works by using each
of the $N$ training examples to ``throw out'' bad hypotheses until
only a small number are left.  But if $\card{\cH} = \infty$, and
you're throwing out a finite number at each step, there will always be
an infinite number remaining.

This means that, if you want to establish sample complexity results
for infinite hypothesis spaces, you need some new way of measuring
their ``size'' or ``complexity.''  A prototypical way of doing this is
to measure the complexity of a hypothesis class as the \emph{number of
  different things it can do}.

As a silly example, consider boolean conjunctions again.  Your input
is a vector of binary features.  However, instead of representing your
hypothesis as a boolean conjunction, you choose to represent it as a
conjunction of \emph{inequalities.}  That is, instead of writing $x_1
\land \lnot x_2 \land x_5$, you write $[x_1 > 0.2] \land [x_2 < 0.77]
\land [x_5 < \pi/4]$.  In this representation, for each feature, you
need to choose an inequality ($<$ or $>$) and a threshold.  Since the
thresholds can be arbitrary real values, there are now infinitely many
possibilities: $\card{\cH} = 2^D \times \infty = \infty$.  However,
you can immediately recognize that on binary features, there really is
no difference between $[x_2 < 0.77]$ and $[x_2 < 0.12]$ and any other
number of infinitely many possibilities.  In other words, \emph{even
  though there are infinitely many hypotheses, there are only finitely
  many behaviors.}

\TODOFigure{thy:vcex}{figure with three and four examples}

The \concept{Vapnik-Chernovenkis dimension} (or \concept{VC
  dimension}) is a classic measure of complexity of infinite
hypothesis classes based on this intuition\sidenote{Yes, this is the
  same Vapnik who is credited with the creation of the support vector
  machine.}.  The VC dimension is a very classification-oriented
notion of complexity.  The idea is to look at a finite set of
unlabeled examples, such as those in Figure~\ref{fig:thy:vcex}.  The
question is: no matter how these points were labeled, would we be able
to find a hypothesis that correctly classifies them.  The idea is that
as you add more points, being able to represent an arbitrary labeling
becomes harder and harder.  For instance, regardless of how the three
points are labeled, you can find a linear classifier that agrees with
that classification.  However, for the four points, there exists a
labeling for which you cannot find a perfect classifier.  The VC
dimension is the \emph{maximum} number of points for which you can
always find such a classifier.

\thinkaboutit{What is that labeling?  What is its name?}

You can think of VC dimension as a game between you and an adversary.
To play this game, \emph{you} choose $K$ unlabeled points however you
want.  Then your adversary looks at those $K$ points and assigns
binary labels to them them however they want.  You must then find a
hypothesis (classifier) that agrees with their labeling.  You win if you
can find such a hypothesis; they win if you cannot.  The VC dimension
of your hypothesis class is the \emph{maximum} number of points $K$ so
that you can always win this game.  This leads to the following formal
definition, where you can interpret \textcolor{darkergreen}{there
  exists} as \textcolor{darkergreen}{your move} and
\textcolor{darkred}{for all} as \textcolor{darkred}{adversary's move}.

\begin{definition} \label{def:thy:vc} For data drawn from some space
  $\cX$, the \textcolor{darkblue}{VC dimension} of a hypothesis space
  $\cH$ over $\cX$ is the \emph{maximal} $K$ such that:
  \textcolor{darkergreen}{there exists} a set $X \subseteq \cX$ of
  size $\card{X}=K$, such that \textcolor{darkred}{for all} binary
  labelings of $X$, \textcolor{darkergreen}{there exists} a function
  $f \in \cH$ that matches this labeling.
\end{definition}

In general, it is much easier to show that the VC dimension is at
least some value; it is much harder to show that it is at most some
value.  For example, following on the example from
Figure~\ref{fig:thy:vcex}, the image of three points (plus a little
argumentation) is enough to show that the VC dimension of linear
classifiers in two dimension is \emph{at least three}.

To show that the VC dimension is \emph{exactly three} it suffices to
show that you \emph{cannot} find a set of four points such that you
win this game against the adversary.  This is much more difficult.  In
the proof that the VC dimension is at least three, you simply need to
provide an example of three points, and then work through the small
number of possible labelings of that data.  To show that it is at most
three, you need to argue that no matter what set of four points you
pick, you cannot win the game.



% VC

% margins

% small norms

% \section{Learning with Noise}

% \section{Agnostic Learning}

% \section{Error versus Regret}


% Despite the fact that there's no way to get better than $20\%$ error
% on this distribution, it would be nice to say that you can still learn
% something from it.  For instance, the predictor that always guesses
% $y=x$ seems like the ``right'' thing to do.  Based on this
% observation, maybe we can rephrase the goal of learning as to find a
% function that does as well as the distribution allows.  In other
% words, on this data, you would hope to get $20\%$ error.  On some other
% distribution, you would hope to get $X\%$ error, where $X\%$ is the
% best you could do.

% This notion of ``best you could do'' is sufficiently important that it
% has a name: the \concept{Bayes error rate}.  This is the error rate of
% the best possible classifier, the so-called \concept{Bayes optimal
%   classifier}.  If you knew the underlying distribution $\cD$, you
% could actually \emph{write down} the exact Bayes optimal classifier
% explicitly.  (This is why learning is uninteresting in the case that
% you know $\cD$.)  It simply has the form:
% %
% \begin{align}
% f^{\text{Bayes}}(\vx) &=
%  \brack{ +1 & \txtif \cD(\vx,+1) > \cD(\vx,-1) \\
%          -1 & \text{otherwise}}
% \end{align}
% %
% The Bayes optimal error rate is the error rate that this
% (hypothetical) classifier achieves:
% %
% \begin{align}
% \ep^{\text{Bayes}} &=
%   \Ep_{(\vx,y) \sim \cD} \big[ y \neq f^{\text{Bayes}}(\vx) \big]
% \end{align}


\section{Further Reading}

TODO



%%% Local Variables:
%%% mode: latex
%%% TeX-master: "courseml"
%%% End:
